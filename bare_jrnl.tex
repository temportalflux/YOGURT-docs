%% bare_jrnl.tex
%% 2018/12/04
%% by Dustin Yost
%%
%% Support sites:
%% http://www.michaelshell.org/tex/ieeetran/
%% http://www.ctan.org/pkg/ieeetran
%% and
%% http://www.ieee.org/

%%*************************************************************************

\documentclass[journal]{IEEEtran}
%
% If IEEEtran.cls has not been installed into the LaTeX system files,
% manually specify the path to it like:
% \documentclass[journal]{../sty/IEEEtran}

% Some very useful LaTeX packages include:
% (uncomment the ones you want to load)


% *** MISC UTILITY PACKAGES ***
%
%\usepackage{ifpdf}
% Heiko Oberdiek's ifpdf.sty is very useful if you need conditional
% compilation based on whether the output is pdf or dvi.
% usage:
% \ifpdf
%   % pdf code
% \else
%   % dvi code
% \fi
% The latest version of ifpdf.sty can be obtained from:
% http://www.ctan.org/pkg/ifpdf
% Also, note that IEEEtran.cls V1.7 and later provides a builtin
% \ifCLASSINFOpdf conditional that works the same way.
% When switching from latex to pdflatex and vice-versa, the compiler may
% have to be run twice to clear warning/error messages.


% *** CITATION PACKAGES ***
%
\usepackage{cite}
% cite.sty was written by Donald Arseneau
% V1.6 and later of IEEEtran pre-defines the format of the cite.sty package
% \cite{} output to follow that of the IEEE. Loading the cite package will
% result in citation numbers being automatically sorted and properly
% "compressed/ranged". e.g., [1], [9], [2], [7], [5], [6] without using
% cite.sty will become [1], [2], [5]--[7], [9] using cite.sty. cite.sty's
% \cite will automatically add leading space, if needed. Use cite.sty's
% noadjust option (cite.sty V3.8 and later) if you want to turn this off
% such as if a citation ever needs to be enclosed in parenthesis.
% cite.sty is already installed on most LaTeX systems. Be sure and use
% version 5.0 (2009-03-20) and later if using hyperref.sty.
% The latest version can be obtained at:
% http://www.ctan.org/pkg/cite
% The documentation is contained in the cite.sty file itself.


% *** GRAPHICS RELATED PACKAGES ***
%
\ifCLASSINFOpdf
  % \usepackage[pdftex]{graphicx}
  % declare the path(s) where your graphic files are
  % \graphicspath{{../pdf/}{../jpeg/}}
  % and their extensions so you won't have to specify these with
  % every instance of \includegraphics
  % \DeclareGraphicsExtensions{.pdf,.jpeg,.png}
\else
  % or other class option (dvipsone, dvipdf, if not using dvips). graphicx
  % will default to the driver specified in the system graphics.cfg if no
  % driver is specified.
  % \usepackage[dvips]{graphicx}
  % declare the path(s) where your graphic files are
  % \graphicspath{{../eps/}}
  % and their extensions so you won't have to specify these with
  % every instance of \includegraphics
  % \DeclareGraphicsExtensions{.eps}
\fi
% graphicx was written by David Carlisle and Sebastian Rahtz. It is
% required if you want graphics, photos, etc. graphicx.sty is already
% installed on most LaTeX systems. The latest version and documentation
% can be obtained at: 
% http://www.ctan.org/pkg/graphicx
% Another good source of documentation is "Using Imported Graphics in
% LaTeX2e" by Keith Reckdahl which can be found at:
% http://www.ctan.org/pkg/epslatex
%
% latex, and pdflatex in dvi mode, support graphics in encapsulated
% postscript (.eps) format. pdflatex in pdf mode supports graphics
% in .pdf, .jpeg, .png and .mps (metapost) formats. Users should ensure
% that all non-photo figures use a vector format (.eps, .pdf, .mps) and
% not a bitmapped formats (.jpeg, .png). The IEEE frowns on bitmapped formats
% which can result in "jaggedy"/blurry rendering of lines and letters as
% well as large increases in file sizes.
%
% You can find documentation about the pdfTeX application at:
% http://www.tug.org/applications/pdftex


% *** MATH PACKAGES ***
%
%\usepackage{amsmath}
% A popular package from the American Mathematical Society that provides
% many useful and powerful commands for dealing with mathematics.
%
% Note that the amsmath package sets \interdisplaylinepenalty to 10000
% thus preventing page breaks from occurring within multiline equations. Use:
%\interdisplaylinepenalty=2500
% after loading amsmath to restore such page breaks as IEEEtran.cls normally
% does. amsmath.sty is already installed on most LaTeX systems. The latest
% version and documentation can be obtained at:
% http://www.ctan.org/pkg/amsmath


% *** SPECIALIZED LIST PACKAGES ***
%
%\usepackage{algorithmic}
% algorithmic.sty was written by Peter Williams and Rogerio Brito.
% This package provides an algorithmic environment fo describing algorithms.
% You can use the algorithmic environment in-text or within a figure
% environment to provide for a floating algorithm. Do NOT use the algorithm
% floating environment provided by algorithm.sty (by the same authors) or
% algorithm2e.sty (by Christophe Fiorio) as the IEEE does not use dedicated
% algorithm float types and packages that provide these will not provide
% correct IEEE style captions. The latest version and documentation of
% algorithmic.sty can be obtained at:
% http://www.ctan.org/pkg/algorithms
% Also of interest may be the (relatively newer and more customizable)
% algorithmicx.sty package by Szasz Janos:
% http://www.ctan.org/pkg/algorithmicx


% *** ALIGNMENT PACKAGES ***
%
%\usepackage{array}
% Frank Mittelbach's and David Carlisle's array.sty patches and improves
% the standard LaTeX2e array and tabular environments to provide better
% appearance and additional user controls. As the default LaTeX2e table
% generation code is lacking to the point of almost being broken with
% respect to the quality of the end results, all users are strongly
% advised to use an enhanced (at the very least that provided by array.sty)
% set of table tools. array.sty is already installed on most systems. The
% latest version and documentation can be obtained at:
% http://www.ctan.org/pkg/array


% IEEEtran contains the IEEEeqnarray family of commands that can be used to
% generate multiline equations as well as matrices, tables, etc., of high
% quality.


% *** SUBFIGURE PACKAGES ***
%\ifCLASSOPTIONcompsoc
%  \usepackage[caption=false,font=normalsize,labelfont=sf,textfont=sf]{subfig}
%\else
%  \usepackage[caption=false,font=footnotesize]{subfig}
%\fi
% subfig.sty, written by Steven Douglas Cochran, is the modern replacement
% for subfigure.sty, the latter of which is no longer maintained and is
% incompatible with some LaTeX packages including fixltx2e. However,
% subfig.sty requires and automatically loads Axel Sommerfeldt's caption.sty
% which will override IEEEtran.cls' handling of captions and this will result
% in non-IEEE style figure/table captions. To prevent this problem, be sure
% and invoke subfig.sty's "caption=false" package option (available since
% subfig.sty version 1.3, 2005/06/28) as this is will preserve IEEEtran.cls
% handling of captions.
% Note that the Computer Society format requires a larger sans serif font
% than the serif footnote size font used in traditional IEEE formatting
% and thus the need to invoke different subfig.sty package options depending
% on whether compsoc mode has been enabled.
%
% The latest version and documentation of subfig.sty can be obtained at:
% http://www.ctan.org/pkg/subfig


% *** FLOAT PACKAGES ***
%
%\usepackage{fixltx2e}
% fixltx2e, the successor to the earlier fix2col.sty, was written by
% Frank Mittelbach and David Carlisle. This package corrects a few problems
% in the LaTeX2e kernel, the most notable of which is that in current
% LaTeX2e releases, the ordering of single and double column floats is not
% guaranteed to be preserved. Thus, an unpatched LaTeX2e can allow a
% single column figure to be placed prior to an earlier double column
% figure.
% Be aware that LaTeX2e kernels dated 2015 and later have fixltx2e.sty's
% corrections already built into the system in which case a warning will
% be issued if an attempt is made to load fixltx2e.sty as it is no longer
% needed.
% The latest version and documentation can be found at:
% http://www.ctan.org/pkg/fixltx2e


%\usepackage{stfloats}
% stfloats.sty was written by Sigitas Tolusis. This package gives LaTeX2e
% the ability to do double column floats at the bottom of the page as well
% as the top. (e.g., "\begin{figure*}[!b]" is not normally possible in
% LaTeX2e). It also provides a command:
%\fnbelowfloat
% to enable the placement of footnotes below bottom floats (the standard
% LaTeX2e kernel puts them above bottom floats). This is an invasive package
% which rewrites many portions of the LaTeX2e float routines. It may not work
% with other packages that modify the LaTeX2e float routines. The latest
% version and documentation can be obtained at:
% http://www.ctan.org/pkg/stfloats
% Do not use the stfloats baselinefloat ability as the IEEE does not allow
% \baselineskip to stretch. Authors submitting work to the IEEE should note
% that the IEEE rarely uses double column equations and that authors should try
% to avoid such use. Do not be tempted to use the cuted.sty or midfloat.sty
% packages (also by Sigitas Tolusis) as the IEEE does not format its papers in
% such ways.
% Do not attempt to use stfloats with fixltx2e as they are incompatible.
% Instead, use Morten Hogholm'a dblfloatfix which combines the features
% of both fixltx2e and stfloats:
%
% \usepackage{dblfloatfix}
% The latest version can be found at:
% http://www.ctan.org/pkg/dblfloatfix


%\ifCLASSOPTIONcaptionsoff
%  \usepackage[nomarkers]{endfloat}
% \let\MYoriglatexcaption\caption
% \renewcommand{\caption}[2][\relax]{\MYoriglatexcaption[#2]{#2}}
%\fi
% endfloat.sty was written by James Darrell McCauley, Jeff Goldberg and 
% Axel Sommerfeldt. This package may be useful when used in conjunction with 
% IEEEtran.cls'  captionsoff option. Some IEEE journals/societies require that
% submissions have lists of figures/tables at the end of the paper and that
% figures/tables without any captions are placed on a page by themselves at
% the end of the document. If needed, the draftcls IEEEtran class option or
% \CLASSINPUTbaselinestretch interface can be used to increase the line
% spacing as well. Be sure and use the nomarkers option of endfloat to
% prevent endfloat from "marking" where the figures would have been placed
% in the text. The two hack lines of code above are a slight modification of
% that suggested by in the endfloat docs (section 8.4.1) to ensure that
% the full captions always appear in the list of figures/tables - even if
% the user used the short optional argument of \caption[]{}.
% IEEE papers do not typically make use of \caption[]'s optional argument,
% so this should not be an issue. A similar trick can be used to disable
% captions of packages such as subfig.sty that lack options to turn off
% the subcaptions:
% For subfig.sty:
% \let\MYorigsubfloat\subfloat
% \renewcommand{\subfloat}[2][\relax]{\MYorigsubfloat[]{#2}}
% However, the above trick will not work if both optional arguments of
% the \subfloat command are used. Furthermore, there needs to be a
% description of each subfigure *somewhere* and endfloat does not add
% subfigure captions to its list of figures. Thus, the best approach is to
% avoid the use of subfigure captions (many IEEE journals avoid them anyway)
% and instead reference/explain all the subfigures within the main caption.
% The latest version of endfloat.sty and its documentation can obtained at:
% http://www.ctan.org/pkg/endfloat
%
% The IEEEtran \ifCLASSOPTIONcaptionsoff conditional can also be used
% later in the document, say, to conditionally put the References on a 
% page by themselves.


% *** PDF, URL AND HYPERLINK PACKAGES ***
%
%\usepackage{url}
% url.sty was written by Donald Arseneau. It provides better support for
% handling and breaking URLs. url.sty is already installed on most LaTeX
% systems. The latest version and documentation can be obtained at:
% http://www.ctan.org/pkg/url
% Basically, \url{my_url_here}.


% *** Do not adjust lengths that control margins, column widths, etc. ***
% *** Do not use packages that alter fonts (such as pslatex).         ***
% There should be no need to do such things with IEEEtran.cls V1.6 and later.
% (Unless specifically asked to do so by the journal or conference you plan
% to submit to, of course. )

% correct bad hyphenation here
\hyphenation{op-tical net-works semi-conduc-tor}




\begin{document}
%
% paper title
% Titles are generally capitalized except for words such as a, an, and, as, at, but, by, for, in, nor, of, on, or, the, to and up, which are usually not capitalized unless they are the first or last word of the title.
% Linebreaks \\ can be used within to get better formatting as desired.
% Do not put math or special symbols in the title.
\title{Finding the User Experience Using YOGURT}
%
%
% author names and IEEE memberships
% note positions of commas and nonbreaking spaces ( ~ ) LaTeX will not break
% a structure at a ~ so this keeps an author's name from being broken across
% two lines.
% use \thanks{} to gain access to the first footnote area
% a separate \thanks must be used for each paragraph as LaTeX2e's \thanks
% was not built to handle multiple paragraphs
%

% TODO: What is my title?
\author{
    Dustin~Yost,~\IEEEmembership{Game Engineer,~Champlain College,}
}

% note the % following the last \IEEEmembership and also \thanks - 
% these prevent an unwanted space from occurring between the last author name
% and the end of the author line. i.e., if you had this:
% 
% \author{....lastname \thanks{...} \thanks{...} }
%                     ^------------^------------^----Do not want these spaces!
%
% a space would be appended to the last name and could cause every name on that
% line to be shifted left slightly. This is one of those "LaTeX things". For
% instance, "\textbf{A} \textbf{B}" will typeset as "A B" not "AB". To get
% "AB" then you have to do: "\textbf{A}\textbf{B}"
% \thanks is no different in this regard, so shield the last } of each \thanks
% that ends a line with a % and do not let a space in before the next \thanks.
% Spaces after \IEEEmembership other than the last one are OK (and needed) as
% you are supposed to have spaces between the names. For what it is worth,
% this is a minor point as most people would not even notice if the said evil
% space somehow managed to creep in.



% The paper headers
\markboth{Finding the User Experience using YOGURT, December 2018}%
{Yost \MakeLowercase{\textit{et al.}}: Finding the User Experience Using YOGURT}
% The only time the second header will appear is for the odd numbered pages
% after the title page when using the twoside option.
% 
% *** Note that you probably will NOT want to include the author's ***
% *** name in the headers of peer review papers.                   ***
% You can use \ifCLASSOPTIONpeerreview for conditional compilation here if
% you desire.



% If you want to put a publisher's ID mark on the page you can do it like
% this:
%\IEEEpubid{0000--0000/00\$00.00~\copyright~2015 IEEE}
% Remember, if you use this you must call \IEEEpubidadjcol in the second
% column for its text to clear the IEEEpubid mark.



% use for special paper notices
%\IEEEspecialpapernotice{(Invited Paper)}




% make the title area
\maketitle

% As a general rule, do not put math, special symbols or citations
% in the abstract or keywords.
\begin{abstract}
This paper introduces YOGURT, a game engine plugin developed in the field of Games User Research. YOGURT was conceptualized, proposed, built, tested, and iterated upon over the course of 4 months. It's purpose is to aid developers in the process of collecting analytical data representing the player experience. YOGURT is also used to then processed said data, and visually represented it in a game engine so that it is easily digestible by developers. The inspiration to develop YOGURT came from a help in the development pipeline, there are few published and easy to access tools which help developers better understand the user experience.
\end{abstract}

% Note that keywords are not normally used for peerreview papers.
\begin{IEEEkeywords}
GUR, games user research, unreal engine, tool development.
\end{IEEEkeywords}


% For peer review papers, you can put extra information on the cover
% page as needed:
% \ifCLASSOPTIONpeerreview
% \begin{center} \bfseries EDICS Category: 3-BBND \end{center}
% \fi
%
% For peerreview papers, this IEEEtran command inserts a page break and
% creates the second title. It will be ignored for other modes.
%\IEEEpeerreviewmaketitle



\section{Introduction}
% The very first letter is a 2 line initial drop letter followed
% by the rest of the first word in caps.
% 
% form to use if the first word consists of a single letter:
% \IEEEPARstart{A}{demo} file is ....
% 
% form to use if you need the single drop letter followed by
% normal text (unknown if ever used by the IEEE):
% \IEEEPARstart{A}{}demo file is ....
% 
% Some journals put the first two words in caps:
% \IEEEPARstart{T}{his demo} file is ....
% 
% Here we have the typical use of a "T" for an initial drop letter
% and "HIS" in caps to complete the first word.
\IEEEPARstart{Y}{our} Own Games User Research Toolset (YOGURT) is a tool developed for the purpose of collecting player experience data, processing it, and visualizing said data in engine. At its core, it's goal is to help game developers better understand the experiences their players have in their game. This article will touch on research of Games User Research field. It will also step through the creation of tool development in Unreal Engine 4, and various takeaways through the development of YOGURT. The most important thing to take away from this journey though, is how programmers can support developers in the effort to better understand the player experience. Why should programmers be involved in this process?

\hfill dty
 
\hfill December 06, 2018

\section{Related Works}

The field of Games User Research (otherwise known as GUR) isn’t exactly a new field in game development. In fact, its been around a while. GUR is the practice of analyzing user experience (UX) for the purpose of improving that same experience. That said, it has only been formally recognized as a discipline within the games industry in the last decade \cite{gur-ch2}. For some studios, user experience research may tend to fall under the umbrella category of quality assurance (QA) testing, as it’s more convenient to test the game for bugs and experience all at once. There is a clear distinction in intent when it comes to QA vs GUR testing. Per definition, Quality Assurance testing is meant to test the quality of a game - to find bugs so that these bugs can be fixed so that the game is more stable. The purpose of Games User Research testing, however, is to collecting, analyzing, and act upon on data pertaining to the user’s experience of a game for the purpose of improving said experience. In short, QA is the practice of making a game bug free, whereas GUR must assume a game is stable and looks at how to give players the most enjoyable experience.
Games User Research can come in many forms. There are some companies like Player Research who have departments solely dedicated to GUR testing and analysis, and have even looked at best ways to set up a GUR lab \cite{gur-ch6}. Many studios use surveys as one way to gather knowledge on the player experience \cite{gur-ch9}. Surveys are by far the easiest to set up, but have pitfalls due to the reliance on player memory and transparency. There many are other ways to observe the player experience; recording video of players as they play the game, keeping track of player comments, the Think Aloud protocol, using other inputs like the Hopson Boxes, or utilizing biometrics. \cite{gur-ch11} \cite{gur-ch12} \cite{gur-ch16} All that said thought, one of the most useful forms of data collection involves analytics/telemetry, and requires minimal or no effort on the part of the player. \cite{gur-ch19} Analytics provide a concrete way for developers to see what players are doing in their game, or what is happening to them, so that as developers, they can make more informed decisions. At its core, this means gathering data, analysis, visualization, reporting, and taking action. \cite{gur-ch19}
Individual companies and engine developers alike have already started integrating analytics into their games and platforms. Unity presented a talk during Unite 2015 which showed off their 3-dimensional heatmapping in engine \cite{unite2015}, not to mention Unity has a whole section of their pipeline dedicated to tracking game analytics. There are individual developers out there who have made analytical tools before, such as Prometheus (a 2D heatmapping tool) \cite{gibbs}. There are also a number of companies which provide game developers with ways to track analytics outside of specific engines. \cite{simpleusability} \cite{drachen} The field of Telemetry/Analytics and Biometrics is not without its case studies and research. A case study by A.Drachen looked at the player experience in platformers, focusing on comparing the difficulty of tasks with the errors made in each task. \cite{wehbe} A paper on the theoretical mathematics behind how to create heatmaps by binning data with respect to the data vs the time it was recorded. \cite{kumatani} Tool suites and studies which explore the best way to interpret player emotional state, using technology such as webcams and bpm. \cite{dingli} A case study explored the best way to evaluate user interfaces using different evaluation methods. \cite{nielsen} In an article about Gameplay Approachability Principles, Heather Desurvire and Charlotte Wiberg compared GAP with usability testing and focuses on finding the best heuristics to evaluate player experience. \cite{desurvire} In a paper on gameplay flow and immersion, Lennart Nacke and Craig A. Lindley explore commonly used, yet vague terms to describe the player experience and how their definitions affect player research. \cite{nacke}

% needed in second column of first page if using \IEEEpubid
%\IEEEpubidadjcol

\section{YOGURT}

The purpose of this study was to explore the GUR discipline, looking at problems and challenges as a programmer. As such, after gathering a fundamental understanding of the field, the next goal was to craft a proposal for a tool which would enter development. This tool eventually became known as Your Own Games User Research Toolset (YOGURT). YOGURT is a Unreal Engine 4 plugin which provides a framework for recording data during gameplay and loading it into the editor at a later point. At its inception, YOGURT was planned to have native support for heat-maps, as they are a frequently used tool for tracking anything from player movement to death rates to actions in the world to player emotional input (lookup Hopson Boxes). YOGURT was also planned to provide native support for a set of features called Timelines. This feature set would allow developers to bin/chunk data according to when they occurred during gameplay so that developers could view specific chunks of gameplay. It would also allow developers to “playback” data from a starting time to an ending time, seeing what players are doing in the environment over the course of gameplay.
Getting an Unreal Engine plugin up and running took a good chunk of time out of the couple of weeks set aside for prototyping. With lots of hard work, and a fair bit of luck, YOGURT took form with basic functionality. It allowed developers to specify what areas to track player position heat-mapping, save that data during the game, and enabled developers to load it into engine to see the heat-map in the 3-dimensional space. Timlines also got partially implemented; data could be binned according to the time span that developers wanted to view.
YOGURT was built alongside a game in active development called Capital Vice. As such, YOGURT was influenced by communication with the development team to determine what kind of GUR tool would be most useful. During planning, this resulted in the goal to use a heat-map to track player position throughout gameplay. Capital Vice was a multi-player networked game, with a playtime of approximately 15 minutes. Once Yogurt was stable enough to start recording data, it was able to gather player telemetry from each play session. Data from these sessions was then used not only relay reports back to the team but fine tune how YOGURT worked and its underlying systems. It is valuable to note that the studio in which YOGURT was developed had no GUR lab. Instead, data was gathered in the QA lab, which functioned as a hybrid for the purpose of finding bugs and getting player feedback on games. This meant that data gathered during testing was not always valuable, and could have be tainted by bugs.
After Yogurt was stable and was able to start gathering data, various bugs were found in how the system processed and rendered data. Over the course of the remaining weeks, YOGURT was iterated upon, bugs patched, and visualization improved.

\section{Discussion}

Working on tools for the Games User Research discipline can be time consuming and difficult. Had this tool been developed in isolation, it would not have been as easy to iterate and find bugs. The reality was though, that YOGURT was developed in tandem with a game in the prototyping phase. As such, it allowed development to be informed by lenses of both data science and game development. This style of development also allowed for easy access to be able to internally adjust the tool as the game team needed to.
Starting this process of iteration as early as possible was also critical to YOGURTS success. Gathering the perspectives and goals of the developer team to get a proof of concept working early can make or break a tool. Actually working with a game team allowed both teams to focus in and cater specific data modules depending on what kind of data would be useful for game iteration. Scope and feature creep exist in this realm, just like any other software or development realm. Some features of YOGURT were cut due to time constraints and bugs, like parts of timelines. As a data scientist, having control of the tool to design, create, and iterate on has been extremely useful. Data science is not the end all be all though. YOGURT would not have been as successful without the skills and insight that come from programmers in the game field.
All that said, the most critical portion to creating any good tool comes down to scalability. When it comes to constructing a GUR tool, it is especially important to be sure that the core of what your working on is modular. Modularity can enable rapid iteration and testing when done right, or cause weeks of delays and refactoring if done wrong. At one point or another, the development team may ask for something you, as a programmer or data scientist, hadn’t thought of. So as a programmer participating and innovating in the GUR field, it is incredibly valuable to have a tool which you can easily add another module to and it still work basically the same.

\section{Future Works}

Looking to the future of Your Own Games User Research Toolset, there are a number of things that can be fine tuned and added that would make the tool more useful. Unfortunately, the code base was not always modular. There were portions of the semester in which crunched crept in, and it made the code behind the tool less malleable. It would be great to expose the system and make it easier to work with, so that new modules and ideas are easier to implement (this is a given for any project though). YOGURT would be a great place for other kinds of modules like static event trackers or storyboards. There is even a place for eye-tracking, to see how players interact with UI in games. For the timelines portion of YOGURT, the playback option for developers to see what players are doing could be huge. Finally, making this all easier to access in Unreal’s SlateUI system, perhaps a new editor window, would do wonders for developer ease of access - especially for small teams and studios.


% An example of a floating figure using the graphicx package.
% Note that \label must occur AFTER (or within) \caption.
% For figures, \caption should occur after the \includegraphics.
% Note that IEEEtran v1.7 and later has special internal code that
% is designed to preserve the operation of \label within \caption
% even when the captionsoff option is in effect. However, because
% of issues like this, it may be the safest practice to put all your
% \label just after \caption rather than within \caption{}.
%
% Reminder: the "draftcls" or "draftclsnofoot", not "draft", class
% option should be used if it is desired that the figures are to be
% displayed while in draft mode.
%
%\begin{figure}[!t]
%\centering
%\includegraphics[width=2.5in]{myfigure}
% where an .eps filename suffix will be assumed under latex, 
% and a .pdf suffix will be assumed for pdflatex; or what has been declared
% via \DeclareGraphicsExtensions.
%\caption{Simulation results for the network.}
%\label{fig_sim}
%\end{figure}

% Note that the IEEE typically puts floats only at the top, even when this
% results in a large percentage of a column being occupied by floats.


% An example of a double column floating figure using two subfigures.
% (The subfig.sty package must be loaded for this to work.)
% The subfigure \label commands are set within each subfloat command,
% and the \label for the overall figure must come after \caption.
% \hfil is used as a separator to get equal spacing.
% Watch out that the combined width of all the subfigures on a 
% line do not exceed the text width or a line break will occur.
%
%\begin{figure*}[!t]
%\centering
%\subfloat[Case I]{\includegraphics[width=2.5in]{box}%
%\label{fig_first_case}}
%\hfil
%\subfloat[Case II]{\includegraphics[width=2.5in]{box}%
%\label{fig_second_case}}
%\caption{Simulation results for the network.}
%\label{fig_sim}
%\end{figure*}
%
% Note that often IEEE papers with subfigures do not employ subfigure
% captions (using the optional argument to \subfloat[]), but instead will
% reference/describe all of them (a), (b), etc., within the main caption.
% Be aware that for subfig.sty to generate the (a), (b), etc., subfigure
% labels, the optional argument to \subfloat must be present. If a
% subcaption is not desired, just leave its contents blank,
% e.g., \subfloat[].


% An example of a floating table. Note that, for IEEE style tables, the
% \caption command should come BEFORE the table and, given that table
% captions serve much like titles, are usually capitalized except for words
% such as a, an, and, as, at, but, by, for, in, nor, of, on, or, the, to
% and up, which are usually not capitalized unless they are the first or
% last word of the caption. Table text will default to \footnotesize as
% the IEEE normally uses this smaller font for tables.
% The \label must come after \caption as always.
%
%\begin{table}[!t]
%% increase table row spacing, adjust to taste
%\renewcommand{\arraystretch}{1.3}
% if using array.sty, it might be a good idea to tweak the value of
% \extrarowheight as needed to properly center the text within the cells
%\caption{An Example of a Table}
%\label{table_example}
%\centering
%% Some packages, such as MDW tools, offer better commands for making tables
%% than the plain LaTeX2e tabular which is used here.
%\begin{tabular}{|c||c|}
%\hline
%One & Two\\
%\hline
%Three & Four\\
%\hline
%\end{tabular}
%\end{table}


% Note that the IEEE does not put floats in the very first column
% - or typically anywhere on the first page for that matter. Also,
% in-text middle ("here") positioning is typically not used, but it
% is allowed and encouraged for Computer Society conferences (but
% not Computer Society journals). Most IEEE journals/conferences use
% top floats exclusively. 
% Note that, LaTeX2e, unlike IEEE journals/conferences, places
% footnotes above bottom floats. This can be corrected via the
% \fnbelowfloat command of the stfloats package.




\section{Conclusion}
Throughout this article, you’ve just barely scratched the surface of GUR. There are thousands of games user researchers out there, some of whom are published and have written articles or books on what GUR is and how its used. You have read through a journey into GUR tool development, starting with the inspirations to create YOGURT, the development process and hiccups along the way, and finally the tool itself and conclusions found from development. YOGURT is just one way to approach Games User Research. It is one approach to supporting developers and data scientists in the GUR field. As programmers, it is important to remain aware of fields tangential to game development, GUR especially. Programmers have the knowledge to be able to make great tools for other developers and data scientists alike, so utilize that knowledge and make some amazing stuff.


% if have a single appendix:
%\appendix[Proof of the Zonklar Equations]
% or
%\appendix  % for no appendix heading
% do not use \section anymore after \appendix, only \section*
% is possibly needed

% use appendices with more than one appendix
% then use \section to start each appendix
% you must declare a \section before using any
% \subsection or using \label (\appendices by itself
% starts a section numbered zero.)
%


%\appendices
%\section{Proof of the First Zonklar Equation}
%Appendix one text goes here.

% you can choose not to have a title for an appendix
% if you want by leaving the argument blank
%\section{}
%Appendix two text goes here.


% use section* for acknowledgment
%\section*{Acknowledgment}


%The authors would like to thank...


% Can use something like this to put references on a page
% by themselves when using endfloat and the captionsoff option.
\ifCLASSOPTIONcaptionsoff
  \newpage
\fi



% trigger a \newpage just before the given reference
% number - used to balance the columns on the last page
% adjust value as needed - may need to be readjusted if
% the document is modified later
%\IEEEtriggeratref{8}
% The "triggered" command can be changed if desired:
%\IEEEtriggercmd{\enlargethispage{-5in}}

% references section

% can use a bibliography generated by BibTeX as a .bbl file
% BibTeX documentation can be easily obtained at:
% http://mirror.ctan.org/biblio/bibtex/contrib/doc/
% The IEEEtran BibTeX style support page is at:
% http://www.michaelshell.org/tex/ieeetran/bibtex/
%\bibliographystyle{IEEEtran}
% argument is your BibTeX string definitions and bibliography database(s)
%\bibliography{IEEEabrv,../bib/paper}
%
% <OR> manually copy in the resultant .bbl file
% set second argument of \begin to the number of references
% (used to reserve space for the reference number labels box)
\begin{thebibliography}{17}

\bibitem{gur-ch9}
F. Bruhlmann, E. Mekler, "Surveys in Games User Research," \emph{Games user research. 1st ed. New York, NY: Oxford University Press}, vol. ED-1, pp. 141-162, Dec. 2018.

\bibitem{desurvire}
Heather Desurvire and Charlotte Wiberg. 2008. Master of the game: assessing approachability in future game design. In \emph{CHI '08 Extended Abstracts on Human Factors in Computing Systems} (CHI EA '08). ACM, New York, NY, USA, 3177-3182. DOI: https://doi.org/10.1145/1358628.1358827

\bibitem{dingli}
A. Dingli, A. Giordimaina and H. P. Martinez, "Experience Surveillance Suite for Unity 3D," \emph{2015 7th International Conference on Games and Virtual Worlds for Serious Applications (VS-Games)(VS-GAMES)}, Skövde, Sweden, 2015, pp. 1-6.
doi:10.1109/VS-GAMES.2015.7295774 keywords:{}, url:doi.ieeecomputersociety.org/10.1109/VS-GAMES.2015.7295774

\bibitem{gur-ch19}
A. Drachen, S. Connor, "Game Analytics for Games User Research," \emph{Games user research. 1st ed. New York, NY: Oxford University Press}, vol. ED-1, pp. 333-353, Dec. 2018.

\bibitem{drachen}
Drachen, A. (2018). \emph{What Is Game Telemetry? - GameAnalytics.} [online] GameAnalytics. Available at: https://gameanalytics.com/blog/what-is-game-telemetry.html [Accessed 7 Dec. 2018].

\bibitem{gibbs}
Justin Gibbs. 2018. \emph{Prometheus}: Games User Research Tool Development Using Best Practices. In \emph{Proceedings of Guildhall Thesis Defense 2018 (Guildhall’18)}. ACM, New York, NY, USA, Article X, 10 pages. https://doi.org/10.475/123\_4

\bibitem{gur-ch12}
T. Knoll, "The think-aloud protocol," \emph{Games user research. 1st ed. New York, NY: Oxford University Press}, vol. ED-1, pp. 189-202, Dec. 2018.

\bibitem{kumatani}
S. Kumatani, T. Itoh, Y. Motohashi, K. Umezu and M. Takatsuka, "Time-Varying Data Visualization Using Clustered Heatmap and Dual Scatterplots," \emph{2016 20th International Conference Information Visualisation (IV)}, Lisbon, Portugal, 2016, pp. 63-68. doi:10.1109/IV.2016.50 keywords:{Data visualization;Space heating;Image color analysis;Clustering algorithms;Correlation;Process control},
url:doi.ieeecomputersociety.org/10.1109/IV.2016.50

\bibitem{gur-ch6}
S. Long, "Designing a Games User Research lab from scratch," \emph{Games user research. 1st ed. New York, NY: Oxford University Press}, vol. ED-1, pp. 81-96, Dec. 2018.

\bibitem{nacke}
Lennart Nacke and Craig A. Lindley. 2008. Flow and immersion in first-person shooters: measuring the player's gameplay experience. In Proceedings of the 2008 Conference on Future Play: Research, Play, Share (Future Play '08). ACM, New York, NY, USA, 81-88. DOI=http://dx.doi.org/10.1145/1496984.1496998

\bibitem{gur-ch16}
L. E. Nacke, "Introduction to Biometric Measures for Games User Research," \emph{Games user research. 1st ed. New York, NY: Oxford University Press}, vol. ED-1, pp. 281-299, Dec. 2018.

\bibitem{nielsen}
Jakob Nielsen and Rolf Molich. 1990. Heuristic evaluation of user interfaces. In Proceedings of the SIGCHI Conference on Human Factors in Computing Systems (CHI '90), Jane Carrasco Chew and John Whiteside (Eds.). ACM, New York, NY, USA, 249-256. DOI=http://dx.doi.org/10.1145/97243.97281

\bibitem{gur-ch11}
M. Sangin, "Observing the player experience," \emph{Games user research. 1st ed. New York, NY: Oxford University Press}, vol. ED-1, pp. 175-188, Dec. 2018.

\bibitem{simpleusability}
Usability Testing and Market Research by SimpleUsability Behavioural Research Consultancy. (2018). \emph{What we do - Usability Testing and Market Research by SimpleUsability Behavioural Research Consultancy.} [online] Available at: http://www.simpleusability.com/our-services/games-testing/ [Accessed 7 Dec. 2018].

\bibitem{unite2015}
YouTube. (2018). \emph{Unite 2015 - Custom Events Best Practices and Introduction to Heatmaps.} [online] Available at: https://youtu.be/aWZiB7jX7C0?t=28m [Accessed 7 Dec. 2018].

\bibitem{wehbe}
Rina R. Wehbe, Elisa D. Mekler, Mike Schaekermann, Edward Lank, and Lennart E. Nacke. 2017. Testing Incremental Difficulty Design in Platformer Games. In \emph{Proceedings of the 2017 CHI Conference on Human Factors in Computing Systems (CHI '17)}. ACM, New York, NY, USA, 5109-5113. DOI: https://doi.org/10.1145/3025453.3025697

\bibitem{gur-ch2}
V. Zammitto, "Games User Research as part of the development process in the game industry," \emph{Games user research. 1st ed. New York, NY: Oxford University Press}, vol. ED-1, pp. 15-30, Dec. 2018.

%\bibitem{wolfshead}
%Wolfsheadonline.com. (2018). \emph{Bushnell's Theorem: Easy to Learn, Difficult to Master | Wolfshead Online.} [online] Available at: http://www.wolfsheadonline.com/bushnells-theorem-easy-to-learn-difficult-to-master/ [Accessed 7 Dec. 2018].

\end{thebibliography}

% biography section
% 
% If you have an EPS/PDF photo (graphicx package needed) extra braces are
% needed around the contents of the optional argument to biography to prevent
% the LaTeX parser from getting confused when it sees the complicated
% \includegraphics command within an optional argument. (You could create
% your own custom macro containing the \includegraphics command to make things
% simpler here.)
%\begin{IEEEbiography}[{\includegraphics[width=1in,height=1.25in,clip,keepaspectratio]{mshell}}]{Michael Shell}
% or if you just want to reserve a space for a photo:

%\begin{IEEEbiography}{Michael Shell}
%Biography text here.
%\end{IEEEbiography}

% if you will not have a photo at all:
%\begin{IEEEbiographynophoto}{John Doe}
%Biography text here.
%\end{IEEEbiographynophoto}

% insert where needed to balance the two columns on the last page with
% biographies
%\newpage

%\begin{IEEEbiographynophoto}{Jane Doe}
%Biography text here.
%\end{IEEEbiographynophoto}

% You can push biographies down or up by placing
% a \vfill before or after them. The appropriate
% use of \vfill depends on what kind of text is
% on the last page and whether or not the columns
% are being equalized.

%\vfill

% Can be used to pull up biographies so that the bottom of the last one
% is flush with the other column.
%\enlargethispage{-5in}



% that's all folks
\end{document}


